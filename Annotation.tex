\documentclass[12pt]{article}
   \usepackage[utf8]{inputenc}
    \usepackage[russian]{babel} 
 
    \usepackage{graphicx}
    \usepackage{geometry}
 \geometry{
 	a4paper,
 	total={170mm,257mm},
 	left=20mm,
 	top=20mm,
 }
    \title{Аннотация}
    \author{А. Третьякова}
    \date{24.04.2020}
 
    \begin{document}


  
Работа посвящена разработке модификаций метода SSA с целью повышения устойчивости к выбросам. 

В работе описана общая схема метода SSA с проекторами на пространство ганкелевых матриц и матриц ранга, не превосходящего r,
где проекторы могут строится по различным нормам. В стандартном варианте SSA используются проекторы по норме в пространстве L2.
Было рассмотрено 2 известных идеи построения устойчивых модификаций. Первая идея --- это замена проекции по норме в пространстве L2 на проекцию по норме в L1. 
Вычисление проектора в L1 на множество матриц ранга, не превосходящего r, является сложной задачей. 
В работе рассмотрен один из вариантов построения данного проектора --- последовательный метод, 
реализованный в R-пакете L1pca. Вторая идея --- использование взвешенной нормы в L2, где точкам, содержащим выбросы, присваивается меньший вес.
Рассмотрен метод с итеративным обновлением весов из статьи [Chen K., Sacchi M., 2015], однако он оказался неподходящим для рядов с нестационарным шумом. Поэтому была 
предложена модификация этого метода, которая расширяет его применимость на случай нестационарного шума. Проведено теоретическое сравнение трудоемкостей
предложенных методов.

Свойства методов были исследованы на численных примерах. Также на основе численных экспериментов были выработаны некоторые рекомендации по применению методов. 

~~~\\
~~~\\


This Graduation Project is devoted to the development of robust modifications of the SSA (Singular Spectrum Analysis) method for time series decomposition. 

In this paper we describe a general scheme of the SSA method with projections on the space of Hankel matrices and matrices of rank not exceeding r,
where the projections can be constructed according to the different norms. The Basic SSA uses L2-projectors. Two well-known ideas for building 
robust modifications were considered.
The first idea is to replace the L2-projection by the L1-projection. The computation of L1-projector onto a set of matrices of rank not exceeding r 
is a complicated task. In this paper one version of the construction of this projector was considered --- an iterative method implemented in the R-package L1pca.
The second idea is to use the weighted norm in L2, where the points with outliers get small weights.
A method with iterative updating of weights from the article [Chen K., Sacchi M., 2015] is considered. However, it turned out to be unsuitable for series with unsteady noise. 
A modification of this method is proposed, which extends its applicability to the case of unsteady noise. A theoretical comparison of computational complexity was also carried out.

The properties of the methods were investigated using numerical examples. Based on numerical experiments,
some recommendations on the application of the methods were developed.
 
    \end{document}